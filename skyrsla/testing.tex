\section{Prófanir}
\LARGE{
1. Basic raspberry pi setup. Við setjum upp einfalda raspberry pi 3 OS sem fylgir með tölvuni.\\

2. Python code í raspberry pi. Setjum upp pycharm og prófum aðeins að forrita á raspberry pi-inu. Bara einhver einföld forrit, til að venjast linux.\\

3. Pinnarnir. Prófum pinnana og hvernig þeir virka með python forritum. Við lærum bara basic "physical computing" og látum python forrit kveikja á peru í gegnum breadboard.\\

4. Motorarnir tengdir í pinna. Við prófum þetta með því að tengja VEX motor í breadboard og svo í 5v pinna. Notum síðan speed controller og kóða fyrir það í python.\\

5. Internet á raspberry pi. Sjáum hvort að við getum sett upp þráðlausa nettenginu við router, og reynum að host-a vefsíðu eða einhvernveginn stream sem við gætum fengið inn á desktop tölvur tengdar við sama router.\\

6. Upload camera feed eitthvert á netið. Sjáum hvort að hægt er að hafa live video feed beint frá vélmenninu og inn á vefsíðu/ipcamerainterface/router, og ef hægt er, reynum að stjórna vélmenninu remotely frá desktop tölvu. Allt gert með python kóða, ef hægt er.\\

7. Upload heat/gas/infrared sensor á netið. Svipað og myndavélin. Tengjum sensors við pinnana og setjum gögnin inn á MySQL database.\\

8. Gyroscope keyrsla með 2 dekkjum. Sjá hvort að hægt sé að halda jafnvægi og keyra vélmennið með aðeins tveimur dekkjum. Ef það er ekki hægt, þá höfum við bara auka dekk. Við prófum þetta í gegnum tengdann gyro og python jafnvægisforrit.\\

9. Piston hopp. Ef við erum með auka tíma, þá rannsökum við hvort að hægt er að tengja gorma við vélmennið og látið vélmennið okkar hoppa. Þurfum að lesa okkur til um þetta, því ég veit sáralítið um gorma.\\
}

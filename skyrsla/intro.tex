\section{Inngangur}
Hér í fyrstu glæru er kynnt fyrir ykkur notkunarmöguleika, fýtusa, forritun og framtíð vélmennisins.
Við Bjartur og Pálmi ætlum okkur til að hanna og smíða lítinn (circa sama stærð og stór kókdós, ef hægt er.) all-terrain áhættu vélmenni sem notar gyroscope og tvö dekk til þess að geta hreyft sig í kringum umhverfið. Til þess að geta nýst sem góður ”bomb-defusal” robot, þá mun vélmennið þurfa skynjara sem myndu nýtast vel í þannig umhverfum, t.d. Infrared, Gas og Hitaskynjara, til þess að átta sig betur á því hvernig svæðið er í kringum það. Við ætlum líka að sjá hvort að hægt er að setja upp nætursjónauka á vélmennið.
Hóparnir sem við hönnum þetta í kringum væru til dæmis Björgunarsveitin, sem gæti notað þetta vélmenni til þess að kíkja niður í hella sem fólk er fast í, eða undir snjó þar sem snjóflóð hafði gerst áður.
Síðan gæti lögreglan notað þetta til þess að komast inn á áhættusvæði þar sem lögreglan myndi helst ekki vilja að senda inn lögreglumann / konu. Slík áhættusvæði gætu t.d. verið þar sem einhver er með byssu eða hníf, eða kannski þar sem einhver væri að halda fólki í gísl.
Vélmennið þyrfti ekki að vera bara á íslandi heldur. SWAT lið í bandaríkjunum gætu notað vélmennið á áhættusvæðum með sömu ástæðum og lögreglan. Það eru náttúrulega miklu meiri líkur að vélmennið væri notað í þeim tilgangi í landi eins og bandaríkjunum frekar en hér, þar sem það er tiltölulega friðsamara hér á landi.
Talandi um það, bandaríski herinn gæti haft afnot af svona vélmennum, þar sem þeir eiga oft í vandræðum með sprengjur sem terroristar gætu haft stillt up einhverstaðar. Áhættusamt er að senda mannfólk inn á þannig svæði þar sem einhver gæti slasast.
Notkunarmöguleikarnir eru ekki bara nothæfir fyrir björgunarsveitir og lögreglur, heldur líka venjulegt fólk, sem gætu notað vélmennið í allskins leitir og ævintýri. T.d. ef einhver áhugasöm manneskja myndi vilja skoða einhverja hella sem manneskja myndi venjulega ekki komast inní, þá væri hægt að senda vélmennið inn. Eða ef einhver myndi missa giftingarhring ofan í holræsi, þá væri hægt að senda vélmennið niður eftir hringnum, til þess að finna hann. 
Við ætlum að nota Raspberri PI vegna þess að Arduino styður live-video-feed myndavélar ekki nógu vel, og Raspberry Pi getur verið með IP camera tengingu í router gegnum mjög einfalda HTML vefsíðu til þess að sjá live-video af því sem vélmennið sér. Vélmennið verður líka að geta sent audio til okkar með myndbandinu. 
Ef við náum að gera þetta fljótlega þá væri frábært að geta gert einhvernveginn piston-operated hoppunarmöguleika fyrir vélmennið, þar sem vélmennið þarf kannski einhverntímann að hoppa upp á borð eða eitthvað. 
Fyrirmyndin okkar er vélmennið í þessu myndbandi frá Boston Dynamics: https://www.youtube.com/watch?v=6b4ZZQkcNEo Við áttum okkur á því að þeir eru með mikið meiri reynslu, og fjármögnun hersins frá bandaríkjunum en okkur langar samt að reyna. Við ætlum að hafa vélmennið á tveimur dekkjum, og nota gyroscope til þess að láta það halda jafnvægi. Ef við náum ekki að láta gyroscope-inn virka þá höfum við bara 1 dekk í viðbót til þess að styðja við í staðinn. 
Forritunarmálið okkar er Python, fyrst að það er þægilegasta og auðveldasta forritunarmálið sem hægt er að nota með Raspberry Pi 3. Við ætlum svo að tengja saman Python úr Raspberry Pi 3 í einfalda mótora sem fá spennu úr breadboard.
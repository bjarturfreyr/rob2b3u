\section{Inngangur}
Við Bjartur og Pálmi ætlum að búa til lítinn og nettann "bomb-defusal" róbot sem notar 2 dekk til að keyra um. Róbotinn verður með marga skynjara m.a. Hita og Gas, og við ætlum að reyna að gera live video-feed frá cameru á róbotinum inn í desktop tölvu. Vélmennið er notað til þess að keyra inn í áhættusvæði sem lögregla/her/björgunarsveit er óviss um hvað er á seyði, og vélmennið sér um að greina svæðið áður en liðið sendir manneskju inn á áhættusvæðið. Við ætlum að nota Raspberri PI vegna þess að Arduino styður live-video-feed myndavélar ekki nógu vel, og Raspberry Pi getur verið með IP camera tengingu í router gegnum mjög einfalda HTML vefsíðu til þess að sjá live-video af því sem vélmennið sér. Vélmennið verður líka að geta sent audio til okkar með myndbandinu. Ef við náum að gera þetta fljótlega þá væri frábært að geta gert einhvernveginn piston-operated hoppunarmöguleika fyrir vélmennið, þar sem vélmennið þarf kannski einhverntímann að hoppa upp á borð eða eitthvað. Fyrirmyndin okkar er vélmennið í þessu myndbandi frá Boston Dynamics: https://www.youtube.com/watch?v=6b4ZZQkcNEo Við áttum okkur á því að þeir eru með hellað military funding frá bandaríkjunum en okkur langar samt að reyna. Við ætlum að hafa vélmennið á tveimur dekkjum, og nota gyroscope til þess að láta það halda jafnvægi. Ef við náum ekki að láta gyroscope-inn virka þá höfum við bara 1 dekk í viðbót til þess að styðja við í staðinn. Forritunarmálið okkar er Python, fyrst að það er þægilegasta og auðveldasta forritunarmálið sem hægt er að nota með Raspberry Pi 3.
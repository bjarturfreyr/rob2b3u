\section{Inngangur}
Hér skal gera lýsingu á verkefninu þ.e hvað,  hvernig og  hvaða forritunarmál, fyrir hverja og hvaða notagildi verkefnið hefur. Minnst 500 orð. Notagildi skiptir miklumáli, reynið að sjá fyrir ykkur hverjir geti notað vélmennið ykkar og í hvaða tilgangi.  Þá kemur í ljós að 500 orð er frekar lítið :-) Hér er gott að byrja á því að lesa til um Arduino en allt hjá þeim er open-sourse og svo er hægt að lesa sér til um efnið í útgefnum bókum sem "programming Arduino \cite{monk} Skoðið vel heimildaskrá og skránna mybib.bib. Hér er gott að lýsa högun kerfisins með orðum og mynd sem þið getið gert í draw.io sjá mynd: 
\begin{figure}[h]
\includegraphics[scale=.3]{img/system}
\end{figure}

\section{Inngangur}
Við Bjartur og Pálmi ætlum að búa til lítinn og nettann "bomb-defusal" róbot sem notar 2 dekk til að keyra um. Róbotinn verður með marga skynjara m.a. Infrared, og við ætlum að reyna að gera live video-feed frá cameru á róbotinum inn í desktop tölvu. Vélmennið er notað til þess að keyra inn í áhættusvæði sem lögregla/her er óviss um hvað er á seyði, og vélmennið sér um að greina svæðið áður en liðið fer inn. Við ætlum að nota Raspberri PI vegna þess að Arduino , og vonandi IP camera tengingu við router í gegnum mjög einfalda HTML vefsíðu til þess að sjá live-video af því sem vélmennið sér. Vélmennið verður líka að geta sent audio til okkar með myndbandinu. Ef við náum að gera þetta fljótlega þá væri frábært að geta gert einhvernveginn piston-operated hoppunarmöguleika fyrir vélmennið, þar sem vélmennið þarf kannski einhverntímann að hoppa upp á borð eða eitthvað.


\begin{figure}[h]
\includegraphics[scale=.3]{img/system}
\end{figure}